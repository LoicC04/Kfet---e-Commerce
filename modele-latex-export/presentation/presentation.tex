\documentclass{polytech-presentation}
\usepackage{pgfpages}
\setbeameroption{show notes on second screen}
\newcommand{\beamerannot}[1]{
    \note{#1}
    %%      \pdfannot {
    %%              /Subtype /Text
    %%              /Name /BeamerNote
    %%              /H Y
    %%              /Contents (#1)
    %%      }
}

\title{Réalisation d'un site web de gestion pour la Kfet}
\author{Loïc Carney et Lacroix Sébastien}
\date{06/05/2012}

\begin{document}

        \frame{\titlepage}

		
	\section<presentation>*{Plan}
            \begin{frame}<beamer>
                \frametitle{Plan}
                \tableofcontents
            \end{frame}

        \section<presentation>{Présentation des objectifs}
        %Seb
            \begin{frame}
                \begin{block}{}

                \end{block}
                \beamerannot{}
            \end{frame}


	\section<presentation>{Le framework Django}%{{{
        %Seb
            \begin{frame}
                \frametitle{Présentation de Django}
                \begin{block}{Django}
                    "The web framework for perfectionists with deadlines." Django est un framework basé sur la structure du MVC.\pause                    
                \end{block}
                \begin{itemize}
                    \item Un langage de template.
                    \item Un contrôleur.
                    \item Une API d'accès aux données.
                \end{itemize}
                \hfill \includegraphics[width=4cm]{logos/pony.jpg}
                \beamerannot{
                    Framework MVT ou MVC\\
                    \begin{itemize}
                        \item Un langage de template : flexible, genere html ou xml
                        \item Un contrôleur : fournis sous forme de remapping d'url avec expressions regulières
                        \item Une API d'accès aux données : Inutile d'ecrire les requetes SQL ...
                    \end{itemize} 
                    Conçu pour les perfectionnistes pressés (slogan)\\
                    Framework Python développé sous licence BSD.\\
                    Forces : Vues génériques (pagination simple, lister objets orga selon date ...), système d'authentification, création de page statique, documentation, gestion des exceptions poussée. Nombreux avantaes autres : serveur web intégré, système de formulaire ...\\
                    Faiblesses : Ajax pas prévu, Migration des modèles.\\
                }
            \end{frame}%}}}

        \section<presentation>{Fonctionnalités à destination des Kfetier}
        %Loic
            \begin{frame}
                \begin{block}{}

                \end{block}
                \beamerannot{}
            \end{frame}

	\section<presentation>{Fonctionnalités de gestion de la Kfet}
        %Loic
            \begin{frame}
                \begin{block}{}

                \end{block}
                \beamerannot{}
            \end{frame}

	\section<presentation>{Expérience utilisateur}
        %Seb
            \begin{frame}
                \begin{block}{}

                \end{block}
                \beamerannot{}
            \end{frame}

	\section<presentation>*{Conclusion}
        %Seb
            \begin{frame}
                \begin{block}{}

                \end{block}
                \beamerannot{}
            \end{frame}




	
\end{document}
