\documentclass{polytech-presentation}
\usepackage{pgfpages}

\title{Réalisation d'un site web de gestion pour la Kfet}
\author{Loïc Carney et Lacroix Sébastien}
\date{06/05/2012}

\begin{document}

        \frame{\titlepage}%{{{

		
	\section<presentation>*{Plan}
            \begin{frame}<beamer>
                \frametitle{Plan}
                \tableofcontents
            \end{frame}%}}}

        \section<presentation>{Présentation des objectifs}%{{{
        %Seb
            \begin{frame}
                \begin{itemize}
                    \item Gestion de l'association ;
                    \item E-commerce complet ;
                    \item Back-office de la Kfet.
                \end{itemize}
                \hfill \includegraphics[width=0.4\linewidth]{logos/logo.png}
              \end{frame}
%}}}

	\section<presentation>{Le framework Django}%{{{
        %Seb
            \begin{frame}
                \frametitle{Présentation de Django}
                \begin{block}{Django}
                    "The web framework for perfectionists with deadlines." Django est un framework basé sur la structure du MVC.\pause                    
                \end{block}
                \begin{itemize}
                    \item Un langage de template ;
                    \item Un contrôleur ;
                    \item Une API d'accès aux données.
                \end{itemize}
                \hfill \includegraphics[width=4cm]{logos/pony.jpg}
                    \end{frame}%}}}

        \section<presentation>{Travail réalisé}%{{{

            \subsection<presentation>{Fonctionnalités à destination des Kfetiers}%{{{
            %Loic
                \begin{frame}
		  \frametitle{Pour les Kfetiers}
		    \begin{block}{Qu'est-ce qu'un Kfetier?}
		     Un Kfetier est un bénévole du club qui vous sert les cafés ou les pizzas pendant les pauses et le
midi.
		    \end{block}

                    \begin{block}{Fonctionnalités}
                     Possède les fonctionnalités d'un utilisateur (étudiant) plus :
		      \begin{itemize}
		       \item Ventes de produits ;
		       \item Gestion des commandes (Encaissement, validation) ;
		       \item Gestion des dettes.
		      \end{itemize}
                    \end{block}
                           \end{frame}
    %}}}

            \subsection<presentation>{Fonctionnalités de gestion de la Kfet}%{{{
            %Loic
                \begin{frame}
		  \frametitle{L'administrateur de la Kfet}
                    \begin{block}{Un administrateur ?}
		      L'administrateur sur le site web représente les personnes à responsabilité pour le club Kfet :
		      \begin{itemize}
		       \item Le président ;
		       \item Le trésorier ;
		       \item Le secrétaire.
		      \end{itemize}
                    \end{block}
                            \end{frame}
                
                \begin{frame}
                 \frametitle{Les fonctionnalités de l'administrateur}
                 \begin{block}{Fonctionnalités}
                  Possède toutes les fonctionnalités d'un Kfetier plus :
                  \begin{itemize}
                   \item Gestion des stocks ;
		   \item Gestion des menus ;
		   \item Gestion des types de paiement ;
		   \item Statistiques de ventes.
                  \end{itemize}
                 \end{block}
                \end{frame}

                %}}}

            \subsection<presentation>{Expérience utilisateur}%{{{
            %Seb
                \begin{frame}
                    \frametitle{Comptes utilisateur}
                    \begin{block}{Gestion de son compte}
                        Un étudiant possédera un compte permettant de commander des pizzas et consulter sa dette. Un enseignant pourra en plus augmenter sa dette.
                    \end{block}
                    \begin{itemize}
                        \item Module complet de gestion de dette ;
                        \item Module complet d'identification.
                    \end{itemize}
                            \end{frame}
                \begin{frame}
                    \frametitle{E-commerce}
                    \begin{block}{Commande des pizzas}
                        Pouvoir commander sa pizza à la pose de 10h30 grâce à son pc portable.
                    \end{block}
                    \begin{itemize}
                        \item Moins de queue, de bouculades ;
                        \item Gestion plus simple pour les Kftier ;
                        \item Module complet d'e-commerce.
                    \end{itemize}
                            \end{frame}
                \begin{frame}
                    \begin{center}
                        Démonstration
                    \end{center}
                \end{frame}%}}}%}}}

	\section<presentation>*{Conclusion}%{{{
        %Seb
            \begin{frame}
                \begin{itemize}
                    \item Compétences techniques acquises :
                        \begin{itemize}
                            \item Python ;
                            \item Django / Développement web ;
                            \item E-commerce.
                        \end{itemize}       
                        ~\\
                    \item Apport à l'école ;
                    \item Finalisation du site à réaliser, bénévolat ; 
                    \item Mise en place.
                \end{itemize}
                         \end{frame}
%}}}



	
\end{document}
