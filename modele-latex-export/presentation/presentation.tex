\documentclass{polytech-presentation}

\title{Promotion Polytech}
\author{Chapi \and Chapo}
\date{Poil au dos.}

\begin{document}

	\frame{\titlepage}
		
	\section<presentation>*{Plan}
		 \begin{frame}<beamer>
	    \frametitle{Plan}
	    \tableofcontents
	  \end{frame}
	
	\section<presentation>*{Classe Beamer.}
		\subsection{Oui alors.}
			\begin{frame}
				\frametitle{D'accord.}
				
				\begin{block}{Polytech}
					Alors, oui, Polytech, ça pr0tche sévère. Et ce, pour trois raisons.
				\end{block}
				
				\begin{itemize}
					\item Parce que déjà, ça pwnz. \pause
					\item Ensuite, parce que ça gruiiik. \pause
					\item Et enfin, parce qu'on est les meilleurs.
				\end{itemize}
				
			\end{frame}
			
			\begin{frame}
				\frametitle{Argument massue}
				
				\begin{block}{Oui, quand même.}
					Ne pas oublier. On envoie du pâté en croûte (vérifier les accents, là, quand même).
				\end{block}
				
				\begin{itemize}
					\item Par contre, les pauses là \pause
					\item Faudrait voir à pas en abuser. \pause
					\item C'est pas que c'est un poil lourd ....
				\end{itemize}
				
			\end{frame}
		
		\subsection{Non franchement}
			\begin{frame}
				\frametitle{Là, on peut plus rien pour vous.}
				
				\begin{block}{Et LaTeX ?}
					Oui, à Polytech, on joue avec du latex. \pause
				\end{block}
				
				Ha, c'te truc d'autiste, quand même.
				
			\end{frame}
		
	
\end{document}
