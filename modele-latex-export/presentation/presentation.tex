\documentclass{polytech-presentation}
\usepackage{pgfpages}
\setbeameroption{show notes on second screen}
\newcommand{\beamerannot}[1]{
    \note{#1}
    %%      \pdfannot {
    %%              /Subtype /Text
    %%              /Name /BeamerNote
    %%              /H Y
    %%              /Contents (#1)
    %%      }
}

\title{Réalisation d'un site web de gestion pour la Kfet}
\author{Loïc Carney et Lacroix Sébastien}
\date{06/05/2012}

\begin{document}

        \frame{\titlepage}%{{{

		
	\section<presentation>*{Plan}
            \begin{frame}<beamer>
                \frametitle{Plan}
                \tableofcontents
            \end{frame}%}}}

        \section<presentation>{Présentation des objectifs}%{{{
        %Seb
            \begin{frame}
                \begin{block}{}

                \end{block}
                \beamerannot{}
            \end{frame}
%}}}

	\section<presentation>{Le framework Django}%{{{
        %Seb
            \begin{frame}
                \frametitle{Présentation de Django}
                \begin{block}{Django}
                    "The web framework for perfectionists with deadlines." Django est un framework basé sur la structure du MVC.\pause                    
                \end{block}
                \begin{itemize}
                    \item Un langage de template.
                    \item Un contrôleur.
                    \item Une API d'accès aux données.
                \end{itemize}
                \hfill \includegraphics[width=4cm]{logos/pony.jpg}
                \beamerannot{
                    Framework MVT ou MVC\\
                    \begin{itemize}
                        \item Un langage de template : flexible, genere html ou xml
                        \item Un contrôleur : fournis sous forme de remapping d'url avec expressions regulières
                        \item Une API d'accès aux données : Inutile d'ecrire les requetes SQL ...
                    \end{itemize} 
                    Conçu pour les perfectionnistes pressés (slogan)\\
                    Framework Python développé sous licence BSD.\\
                    Forces : Vues génériques (pagination simple, lister objets orga selon date ...), système d'authentification, création de page statique, documentation, gestion des exceptions poussée. Nombreux avantaes autres : serveur web intégré, système de formulaire ...\\
                    Faiblesses : Ajax pas prévu, Migration des modèles.\\
                }
            \end{frame}%}}}

        \section<presentation>{Travail réalisé}%{{{

            \subsection<presentation>{Fonctionnalités à destination des Kfetier}%{{{
            %Loic
                \begin{frame}
                    \begin{block}{}

                    \end{block}
                    \beamerannot{}
                \end{frame}
    %}}}

            \subsection<presentation>{Fonctionnalités de gestion de la Kfet}%{{{
            %Loic
                \begin{frame}
                    \begin{block}{}

                    \end{block}
                    \beamerannot{}
                \end{frame}%}}}

            \subsection<presentation>{Expérience utilisateur}%{{{
            %Seb
                \begin{frame}
                    \begin{block}{Gestion de son compte}
                        Un étudiant possédera un compte permettant de commander des pizza et consulter sa dette. Un enseignant pourra en plus augmenter sa dette.
                    \end{block}
                    \begin{itemize}
                        \item Module complet de gestion de dette.
                        \item Module complet d'identification.
                    \end{itemize}
                    \beamerannot{
                        \begin{itemize}
                            \item Module dette : ajout de dette, retirer dette, historique et avertissement dans le futur. Distinction prof étudiant à ajouter. Actuellement Ajout de dette possible et retrait aussi par les kftier.
                            \item Module auth : fournis par django, simple d'utilisation, permet d'avoir différents niveaux de rôles, extenction du modèle pour ajouter des informations sur l'user.
                        \end{itemize}
                    }
                \end{frame}
                \begin{frame}
                    \begin{block}{Commande des pizzas}
                        Pouvoir commander sa pizza à la pose de 10h30 grâce à son pc portable.
                    \end{block}
                    \begin{itemize}
                        \item Moins de queue, de bouculades.
                        \item Gestion plus simple pour les Kftier.
                        \item Module complet d'ecommerce
                    \end{itemize}
                    \beamerannot{                
                        \begin{itemize}
                            \item 15 pizza : 10 web + 5 comptoir
                            \item Pas de papier à recopier ou moins, le cuistot peut débuter direct, gestion informatique des commandes, qui a pris quoi ou pas
                            \item Panier, validation de commande, paiement de la commande (direct via dette ou moneo au comptoir)
                        \end{itemize}                        
                    }
                \end{frame}
                \begin{frame}
                    \begin{center}
                        Démonstration
                    \end{center}
                \end{frame}%}}}%}}}

	\section<presentation>*{Conclusion}%{{{
        %Seb
            \begin{frame}
                \begin{itemize}
                    \item Compétences techniques acquises :
                        \begin{itemize}
                            \item Python
                            \item Django / Développement web
                            \item E-commerce
                        \end{itemize}       
                        ~\\
                    \item Apport à l'école.
                    \item Finalisation du site à réaliser, bénévolat ?
                    \item Mie en place.
                \end{itemize}
                \beamerannot{
                    \begin{itemize}
                        \item Dev python, langage de plus en plus utilisé, offre d'emploi en rapport. Dev Web reelle expérience d'e-commerce.
                        \item Ex trésorier, finalité en tant que kftier
                        \item Certains fonctionnalités encore à écrire, bénévolat cette été pour le terminer.
                        \item Serveur OVH payé par la KFet ? Serveur à l'école/université ? Acclimatation des utilisateurs.                            
                    \end{itemize}

                    $\Rightarrow$ Reste beaucoup à faire mais il y a avait beaucoup à faire !
                }
            \end{frame}
%}}}



	
\end{document}
