\documentclass[twoside,UTF8]{EPURapport}
\input{include.tex}

\thedocument{Rapport de projet}{Développement d'un site de vente et de gestion pour la Kfet de Polytech}{Site web Kfet}

\grade{Département Informatique\\ 5\ieme{} année\\ 2011 - 2012}

\authors{%
	\category{Étudiants}{%
		\name{Loïc CARNEY} \mail{loic.carney@etu.univ-tours.fr}
		\name{Sébastien LACROIX} \mail{sebastien.lacroix@etu.univ-tours.fr}
	}
	\details{DI5 2008 - 2009}
}

\supervisors{%
	\category{Encadrants}{%
		\name{Alexandre LISSY} \mail{alexandre.lissy@univ-tours.fr}
	}
	\details{Université François-Rabelais, Tours}
}

\abstracts{Description en français}
{Mots clés français}
{Description en anglais}
{Mots clés en anglais}

\begin{document}
\renewcommand{\labelitemi}{\textbullet}

\chapter{Introduction}%{{{

    \paragraph{}Dans le cadre de notre dernière année d'étude à Polytech'Tours, nous devions réaliser un projet lié aux technologies web. Nous avons décider de proposer notre propre sujet. L'un de nous ayant été trésorier à la Kfet du département Informatique de l'école, nous étions au courant que toute la gestion de la Kfet se fait sur papier et manuellement. Aucune gestion automatique n'est en place et aucune gestion des stocks n'est effectuée pour le moment. Nous avons donc proposer comme sujet : la réalisation d'un site web permettant de répondre à tous les besoins de gestion de la Kfet: en commençant par la vente jusqu'à la gestion des stocks et en passant par la gestion des dettes.

    \paragraph{}Pour réaliser le projet web, nous avons décidé d'utiliser une technologie web que nous ne connaissions
pas : le  framework web Django qui est basé sur le langage interprété Python. Le but de Django est de donner au
développeur web des outils simples et rapides, en effet son slogan est "Le framework web pour les perfectionnistes sous
pression".

    \paragraph{}La Kfet de Polytech est ouverte à toutes les pauses. Lors des pauses d'inter-cours elle vend des en-cas, confiseries et des boissons chaudes,et lors de la pause de midi elle sert des plats chauds. Les stocks sont donc assez conséquents et varient fortement, la problématique de la gestion des stocks est un des éléments qui nous a incité à réaliser ce projet.

    \paragraph{}Nous allons donc vous présenter tout d'abord le projet au travers du cahier des charges puis nous détaillerons le fonctionnement de chaque module mis en place.
%}}}

%%%%%%%%%%%%%%%%%%%%%%%%%%%%%%%%%%%%%%%%%%%%%%%%%%%%%%%%%%%%%%%%%%%%%%%%%%%%%%%%%%%%%%%%%%%%%%%%%%%%ù
%%%%%%%%%%%%%%%%%%%%%%%%%%%%%%%%%%%%%%%%%%%%%%%%%%%%%%%%%%%%%%%%%%%%%%%%%%%%%%%%%%%%%%%%%%%%%%%%%%%%ù
%%%%%%%%%%%%%%%%%%%%%%%%%%%%%%%%%%%%%%%%%%%%%%%%%%%%%%%%%%%%%%%%%%%%%%%%%%%%%%%%%%%%%%%%%%%%%%%%%%%%ù
\chapter{Cahier des charges}%{{{
% TODO Seb

    \paragraph{}Nous allons dans cette partie définir les fonctionnalités attendues par la Kfet afin de simplifier la vie de ses membres. Nous allons pour cela décomposer les fonctionnalités par rôles de l'association. 
    \paragraph{}Nous allons donc dans un premier temps définir les fonctionnalités de gestion d'association, puis les fonctionnalités d'un kftier vendeur, enfin nous définirons les fonctionnalités proposés à un étudiant non membre de l'association.

    \section{Fonctionnalités de gestion}

        \paragraph{}Le but premier de ce projet était de remettre en place une réelle gestion au sein de la Kfet. Nous voulions ainsi permettre de suivre en détails les stocks, les ventes de la journée ou du mois, le chiffre d'affaire ...

        \subsection{Gestion des stocks}

            \paragraph{}Le manque de gestion administrative de la Kfet entraîne une faiblesse majeure : presque aucune
gestion des stocks n'est assurée. Cette faiblesse se ressent le plus lors du manque de pizza le midi et le
mécontentement des clients qui en découle.
            \paragraph{}Nous nous proposons donc de mettre en place une gestion des stocks complète apportant les fonctionnalités suivantes :\\
            \begin{itemize}
                \item Ajout de fournisseur : Pouvoir avoir plusieurs fournisseur différents, notamment afin de pouvoir différencier les commandes ;\\
                \item Commande de produit : Pouvoir avoir une trace de la commande effectué auprès du fournisseur et ne la valider qu'une fois arrivée. Nous voulons ainsi différencier les produits en stocks, des commandes ;\\
                \item Notification de fin de stock : Recevoir une notification par mail lorsqu'un stock atteint le seuil limite afin de pouvoir réapprovisionner avant la rupture.\\
            \end{itemize}

        \subsection{Gestion des finances}

            \paragraph{}Nous voulions apporter à la Kfet des outils de gestion permettant d'avoir des informations sur les ventes, le chiffre d'affaire ... Ces outils pourrons par exemple permettre de mettre en avant des pertes trop importantes sur certains produits. Nous avons donc définis les fonctionnalités suivantes :\\

            \begin{itemize}
                \item Statistiques de ventes : afficher le chiffre d'affaire, les ventes réalisées sur une période définie ;\\
                \item Gestion des dettes : pouvoir gérer les dettes de tous les clients de façon informatique (enseignants et étudiants) et pouvoir les avertir lorsqu'un plafond est atteint.
            \end{itemize}

    \section{Fonctionnalités de vente}
        
        \paragraph{}Afin de simplifier la gestion de stock et le fonctionnement de la Kfet nous avons voulu proposer des fonctionnalités de ventes. 

        \paragraph{}La partie vente se décompose en deux types : les commandes de menus le midi, et les ventes individuelles.

        \subsection{Gestion des commandes}
            \paragraph{}Les fonctionnalités permettant de gérer les commandes vont permettre de supprimer tous le système complexe et non productif des papiers de commande. Nous proposons donc les fonctions suivantes :
        
            \begin{itemize}
                \item Gestion de commandes : validation, paiement, annulation ;\\                
                \item Historique permettant un calcul de chiffre d'affaire.
            \end{itemize}

        \subsection{Ventes}
            \paragraph{}Afin de pouvoir avoir une gestion des stocks complète il faut qu'une chaque vente soit enregistrée de manière informatique. Pour cela nous proposons de mettre en place une interface facilement adaptable à un appareil tactile permettant de vendre à l'unité ou en lot et permettant de mettre la vente en dette. Une gestion avancée de celles-ci pourra de plus être mise en place permettant d'annuler la vente en cas d'erreur par exemple.

    \section{Fonctionnalités utilisateur}

        \paragraph{}Nous avons voulu proposer des fonctionnalités simplifiant la gestion de la Kfet, mais nous voulons aussi proposer des fonctionnalités intéressantes aux clients.

        \paragraph{}Nous proposons pour cela de mettre en place un site de e-commerce permettant de réaliser n'importe quelle commande en avance. Cela évitera, par exemple, dans le cas de la vente de pizza la cohue au comptoir due au nombre limité de commande.

        \paragraph{}Un compte utilisateur permettra de plus de pouvoir consulter sa dette, voire de la régler par carte bleue ou paypal.
        %}}}

%%%%%%%%%%%%%%%%%%%%%%%%%%%%%%%%%%%%%%%%%%%%%%%%%%%%%%%%%%%%%%%%%%%%%%%%%%%%%%%%%%%%%%%%%%%%%%%%%%%%ù
%%%%%%%%%%%%%%%%%%%%%%%%%%%%%%%%%%%%%%%%%%%%%%%%%%%%%%%%%%%%%%%%%%%%%%%%%%%%%%%%%%%%%%%%%%%%%%%%%%%%ù
%%%%%%%%%%%%%%%%%%%%%%%%%%%%%%%%%%%%%%%%%%%%%%%%%%%%%%%%%%%%%%%%%%%%%%%%%%%%%%%%%%%%%%%%%%%%%%%%%%%%ù
\chapter{Le framework Django}

\section{Présentation}
Afin de réaliser notre site web, nous avons choisi d'utiliser Django. Django est un framework web Python de haut niveau
qui a comme slogan :\\
\begin{center}
    \begin{LARGE}
    The web framework for perfectionnists with deadlines
    \end{LARGE}
\end{center}

Cela signifie : "Le framework web pour les perfectionnistes sous pression". Ce qui nous correspond parfaitement.

\begin{figure}[H]
 \centering
 \includegraphics[width=0.5\linewidth]{./logos/pony.jpg}
 \caption{Le logo représentatif de Django}
\end{figure}

Le framework Django est composé d'une multitude d'outils très puissant. De plus il est basé sur le modèle de framework MVT :
\begin{itemize}
    \item Modèle : C'est ici que nous définiront les objets que nous manipulons sur le site, en l'occurrence nos objets de base de données ;
    \item Vue : Permet de manipuler les objets de la base de données et les envoies aux pages HTML qui sont générées en fonction de la demande de l'utilisateur ;
    \item Template : Spécifie la structure de la page HTML. On y indique commet et où afficher les variables sur la page.
\end{itemize}

\paragraph{Pourquoi avoir choisir Django ?} Étant en école d'ingénieurs, il est très important d'avoir la volonté d'apprendre constamment de nouvelles technologies. Et cette année, nous avons entendu parlé du langage Python. Python est un langage très simple et très facile à apprendre.



\section{Outils très utiles}
Django possède toute une panoplie d'outils très utiles. Voici quelques uns des outils que nous avons utilisé :
\begin{itemize}
    \item Moteur de réécriture d'URL ;
    \item Moteur de template ;
    \item Serveur web de développement ;
    \item ORM compatible avec la base de données MySQL ;
    \item Moteur de formulaire ;
    \item Interface d'administration.
\end{itemize}

Dans cette section sera décrit brièvement chacun des outils afin de comprendre la facilité d'usage de Django. On peut retrouver facilement sur le site officiel de Django \url{https://www.djangoproject.com/} toute la documentation ainsi que des tutoriels permettant d'apprendre son fonctionnement.

    \subsection{Moteur de réécriture d'URL}
Afin d'accéder à toutes les pages, Django possède un moteur d'analyse et de réécriture des URL. En effet, dans chaque module du site web, un fichier "url.py" permet d'associer une URL et une fonction. À chaque fois qu'un utilisateur demande une URL qui est spécifier dans ce fichier, alors le serveur web appelle la fonction que lui a associé.

    \subsection{Moteur de template}
Afin d'afficher des pages HTML, Django possède un moteur de template. Les templates sont en réalité des pages HTML possédant des instructions spéciales que seul le moteur de template comprend.

\lstset{language=HTML}
\begin{lstlisting}
<h1>{{ title }}</h1>
\end{lstlisting}

La variable "title" est une variable de contexte. Chaque page générée par Django possède son propre contexte. Le contexte est en réalité un ensemble de variables qui ont été définis auparavant dans la fonction qui a été appelée par l'utilisateur (L'utilisateur a fait appel à la fonction à travers l'URL).

On peut rapprocher ce fonctionnement de celui d'une page PHP classique à ceci près que le traitement et la récupération des variables se fait dans le fichier de vue.

    \subsection{Serveur de développement web}
Le langage Python possède une bibliothèque permettant de créer un serveur HTTP. Django a repris cet outil simple et puissant et l'a transformé afin d'en faire son propre serveur qui dispose de toutes les fonctionnalités que Django peut proposer.

    \subsection{ORM}
Object-Relational Mapping est une technique de programmation permettant de communiquer avec une base de données. On manipule les entrées d'une base de données relationnelle comme s'ils étaient des objets.

Grâce à cette technique, on ne s'embarrasse pas d'écrire nos requêtes SQL. On manipule donc des objets que l'on appelle
"modèle".

\lstset{language=Python}
\begin{lstlisting}
from django.db import models
class Produit(models.Model):
        nom = models.CharField(max_length=200)
                prix = models.DecimalField(max_digits=10, decimal_places=2)
\end{lstlisting}

Puis, si on a besoin des entrées de cet objet de la base de données :

\begin{lstlisting}
Produit.objects.all()
\end{lstlisting}

Cette ligne permet de récupérer au sein d'une toutes les entrées "Panier".

    \subsection{Moteur de formulaire}
Une fonctionnalité très pratique de Django est la génération automatique de formulaire.

\begin{lstlisting}
class CreationProduitForm(forms.ModelForm):
    class Meta:
        model = Produit
        exclude=('fournisseur', 'quantite', 'quantiteCommandeFournisseur')

    categorie = forms.ModelChoiceField(Categorie.objects, widget=SelectWithPopUp)
    articleDouble = forms.BooleanField(required=False, label="Article compte double dans un menu ?")
\end{lstlisting}

On créé donc une nouvelle classe en lui indiquant qu'on souhaite créer un formulaire à partir de la classe "Produit". On peut lui préciser d'exclure certains attributs et en personnaliser d'autres.

Ces quelques lignes ne créent pas seulement un formulaire mais aussi tout ce qui va avec : validation des données, fonctions de vérifications de type(si on demande un entier, une chaîne de caractères, une adresse mail), \ldots.

    \subsection{Interface d'administration}
Habituellement, quand on développe un site web on passe toujours du temps à réaliser une interface d'administration afin de pouvoir créer les objets de la base de données. Django propose une interface d'administration générée automatiquement. On peut y spécifier chacun des objets que l'on souhaite voir apparaître dans l'interface d'administration.

Par exemple, si on souhaite ajouter/modifier/supprimer un produit à partir de l'interface d'administration, il suffit de déclarer l'objet dans le fichier "admin.py".

\begin{figure}[H]
    \centering
    \includegraphics[width=\linewidth]{./logos/admin.png}
    \caption{L'interface d'administration}
\end{figure}

%%%%%%%%%%%%%%%%%%%%%%%%%%%%%%%%%%%%%%%%%%%%%%%%%%%%%%%%%%%%%%%%%%%%%%%%%%%%%%%%%%%%%%%%%%%%%%%%%%%%ù
%%%%%%%%%%%%%%%%%%%%%%%%%%%%%%%%%%%%%%%%%%%%%%%%%%%%%%%%%%%%%%%%%%%%%%%%%%%%%%%%%%%%%%%%%%%%%%%%%%%%ù
%%%%%%%%%%%%%%%%%%%%%%%%%%%%%%%%%%%%%%%%%%%%%%%%%%%%%%%%%%%%%%%%%%%%%%%%%%%%%%%%%%%%%%%%%%%%%%%%%%%%ù
\chapter{Module Compte}
% TODO Seb

    \paragraph{}Afin de pouvoir mettre en place ce site de gestion il nous fallait une gestion complète d'utilisateurs afin pouvoir définir les rôles de chacun. Django propose une gestion utilisateur permettant la gestion des rôles, des accès ... Nous allons donc détailler les fonctionnalités de ce module.

    \section{Modèle de base de données}

        \paragraph{}Django fournit une table utilisateur avec les attributs suivants par défaut :\\
        \begin{itemize}
            \item username : Login utilisateur ;\\
            \item first\_name : Prénom de l'utilisateur ;\\
            \item last\_name : Nom de famille de l'utilisateur ;\\
            \item email : Adresse email de l'utilisateur ; \\
            \item password : Mot de passe crypté ;\\
            \item is\_staff : Permet de définir le rôle de staff ;\\
            \item is\_superuser : Permet de définir le rôle d'administrateur ;\\
            \item last\_login : Dernière visite de l'utilisateur ;\\
            \item date\_joined : Date de création de l'utilisateur.\\
        \end{itemize}

        \paragraph{}Afin de compléter ce modèle fournis par défaut nous avons mis en place une table UserProfile. Nous
avons ainsi pu ajouter les attributs permettant de connaître la promotion, le panier, et la dette d'un utilisateur.

    \section{Création de compte}

        \paragraph{}Afin de créer la page de création de compte nous avons utilisé Django qui nous permet de créer des formulaires complet à partir d'un modèle de base de données. Django permet notamment d'implémenter un formulaire vérifiant la validité des champs.

        \paragraph{}Lors de la validation du formulaire les champs sont vérifiés afin de tenir compte de plusieurs contraintes. Le nom d'utilisateur ne doit pas déjà être présent dans la base de données, de même pour l'adresse email. Les deux champs mot de passe doivent de plus correspondre. Si ces contraintes sont vérifiées une entrée dans la base de données est créé pour cet utilisateur.

    \section{Identification et gestion de comptes}

        \paragraph{}Un utilisateur, une fois inscrit, peut s'identifier sur le site. L'identification est conservée sous la forme d'une session et cette identification est vérifiée pour afficher les pages autorisées pour cet utilisateur.

        \paragraph{}L'utilisateur a de plus une âge gestion de comptes lui fournissant les informations stockées le
concernant. Cette page contient notamment le montant de sa dette actuelle et l'historique de ses commandes, quelles que
soit leurs statut.

    \section{Travail restant}
        
        \paragraph{}Il nous reste à mettre en place une fonction permettant de rappeler son mot de passe à un utilisateur lors de l'oubli de celui-ci. Il faudrait de plus que l'utilisateur puisse modifier ses informations.

        \paragraph{}Il faudrait de plus réaliser une partie administration des comptes permettant de désactiver, supprimer des comptes et d'élever ou réduire les droits d'un utilisateur.

%%%%%%%%%%%%%%%%%%%%%%%%%%%%%%%%%%%%%%%%%%%%%%%%%%%%%%%%%%%%%%%%%%%%%%%%%%%%%%%%%%%%%%%%%%%%%%%%%%%%ù
%%%%%%%%%%%%%%%%%%%%%%%%%%%%%%%%%%%%%%%%%%%%%%%%%%%%%%%%%%%%%%%%%%%%%%%%%%%%%%%%%%%%%%%%%%%%%%%%%%%%ù
%%%%%%%%%%%%%%%%%%%%%%%%%%%%%%%%%%%%%%%%%%%%%%%%%%%%%%%%%%%%%%%%%%%%%%%%%%%%%%%%%%%%%%%%%%%%%%%%%%%%ù
\chapter{Module Commandes}
% TODO Seb

    \paragraph{}Ce module va apporter des fonctionnalités à la fois aux utilisateurs et aux kftier préparant les commandes.

    \section{Partie Utilisateur}

        \paragraph{}Un utilisateur pourra commander son menu pour le midi afin d'éviter de faire la queue. Cette commande se déroulera en plusieurs étapes : choix des produits, choix des menus, validation du panier.

        \subsection{Produits}

            \subsubsection{Fonctionnalités}

                \paragraph{}Un utilisateur pourra choisir parmi des catégories définies par l'administrateur. Une fois la catégorie choisie il aura le choix entre plusieurs produits. Il pourra alors directement commander un produit ou obtenir des détails sur celui-ci en le sélectionnant. Parmi ces détails des commentaires sont présents, postés par les utilisateurs, supprimable par leurs auteurs ou l'administrateur afin d'éviter les propos déplacés.

                \paragraph{}Enfin pour terminer sur l'aspect visuel présenté à l'utilisateur nous avons mis en place des publicités mettant en valeurs certains produits.            

                \begin{center}
                    \includegraphics[width=0.8\linewidth]{logos/commande.png}
                \end{center}

            \subsubsection{Développement}

                \paragraph{}La liste des produits et des catégorie sont réalisées à partir d'une requête. Dans le cas des catégorie elle est réalisé lors du chargement de toutes les pages via un contextProcessor. Au contraire la liste des produits est réalisée une seule fois mais est paginée permettant d'avoir une page d'une hauteur convenable.

                \paragraph{}Le détail d'un produit est réalisé à partir d'un requête sur ce produit. Les commentaire sont sauvegardés dans une table séparée relié par une clé étrangère au produit lui correspondant. Une vérification d'utilisateur est réalisée afin que seuls les auteurs d'un commentaire puissent le supprimer.

                \paragraph{}Lorsqu'un produit est ajouté au panier, une nouvelle entrée produit\_panier est créé, reliée à un panier, lui-même relié à un utilisateur. Ces produits panier vont permettre de mettre en place le panier de l'utilisateur.

        \subsection{Menus}

            \subsubsection{Fonctionnalités}

                \paragraph{}Les menus permettent de réunir plusieurs articles et d'obtenir une réduction sur le total de ceux-ci. L'utilisateur, via un choix de menu, choisir les différentes composantes de celui-ci. Il sera ensuite ajouté au panier.

                \paragraph{}Le CSS de cette fonctionnalité est à finaliser afin de rendre plus ergonomique la sélection des différents articles.

            \subsubsection{Développement}
                
                \paragraph{}Un menu sera ajouté dans la base de donnée sous la forme d'une entrée regroupant les différentes composantes de celui-ci. Différentes contraintes sont mises en place afin de limiter dans le choix des articles. Un menu correspond ainsi aux attentes des administrateurs de la Kfet.

                \paragraph{}Lors de l'ajout au panier d'un menu, il est considéré comme un produit à part mais est tout de même pris en compte dans le total.            

        \subsection{Panier}

            \subsubsection{Fonctionnalités}
                
                \paragraph{}Le panier apporte les fonctionnalités de base à l'utilisateur : suppression, ajout, modification de la quantité. Il permet de choisir son mode de paiement et enfin de le valider. Une fois une commande validée par l'utilisateur, elle est mise en attente de validation par un Kftier.

                \paragraph{}Un utilisateur pourra suivre l'état et l'historique de ses commandes dans son compte utilisateur.

            \subsection{Développement}

                \paragraph{}Au niveau de la validation du panier, un total de celui-ci est réalisé avant sa validation. Dans le cas d'une dette la dette est automatiquement ajoutée à l'utilisateur correspondant via une requête.

                \paragraph{}Différentes requêtes permettent de modifier les entrées du panier.

    \section{Partie Kftier}

        \paragraph{}Une fois une commande validée un Kftier va la prendre en charge et la réaliser. Il va pour cela modifier son statut pour "En cours".

        \paragraph{}Afin de gérer les commandes un Kftier aura un accès à un menu lui listant les différentes commandes en cours, validée ou annulée. Il pourra ainsi modifier l'état des commandes en fonction de leurs évolutions.

        \begin{center}
            \includegraphics[width=0.8\linewidth]{logos/gestionCommande.png}
        \end{center}


%%%%%%%%%%%%%%%%%%%%%%%%%%%%%%%%%%%%%%%%%%%%%%%%%%%%%%%%%%%%%%%%%%%%%%%%%%%%%%%%%%%%%%%%%%%%%%%%%%%%ù
%%%%%%%%%%%%%%%%%%%%%%%%%%%%%%%%%%%%%%%%%%%%%%%%%%%%%%%%%%%%%%%%%%%%%%%%%%%%%%%%%%%%%%%%%%%%%%%%%%%%ù
%%%%%%%%%%%%%%%%%%%%%%%%%%%%%%%%%%%%%%%%%%%%%%%%%%%%%%%%%%%%%%%%%%%%%%%%%%%%%%%%%%%%%%%%%%%%%%%%%%%%ù
\chapter{Module Gestion Stocks}
% TODO Loïc
\label{stock}

Le module de gestions des stocks, comme son nom l'indique, permet de gérer les stocks de la Kfet. Ce module a été en grande partie développer au cours des TP de la matière "Technologie Web Avancée".

En plus de gérer les stocks de tous les produits, c'est dans ce module que l'on peut :
\begin{itemize}
    \item Créer/modifier/supprimer des fournisseurs ;
    \item Attacher des produits à un fournisseur (création/modification/suppression) ;
    \item Réapprovisionner les produits.
\end{itemize}

Comme on peut le deviner, nous avons choisi d'attacher un produit à un fournisseur. Donc un fournisseur possède 0 ou plusieurs produits, c'est une façon de bien différencier les produits et aussi de pourvoir dans le futur enregistrer les commandes que l'on effectue au sein même de notre application web.

Pour l'instant, notre gestion de stock permet de réapprovisionner les stocks mais n'enregistre pas les commandes fournisseurs que l'on peut faire. Cette fonctionnalité sera développé dans le futur.

\section{Fonction du module}
    \subsection{Modèle de base de données}
\paragraph{Un fournisseur}
\begin{itemize}
    \item Nom ;
    \item Adresse ;
    \item Téléphone ;
    \item Mail ;
    \item Description ;
\end{itemize}

\paragraph{Un produit}
\begin{itemize}
        \item nom ;
        \item prix ;
        \item articleDouble ;
        \item info ;
        \item ingredients ;
        \item categorie : Chaque produit appartient à une catégorie (Pizza, Sandwich, Boisson, \ldots);
        \item image : Chaque produit possède son image afin d'être visible sur le site ;
        \item quantite : Quantité en stock;
        \item quantiteCommandeFournisseur : Quantité actuellement en commande ;
        \item fournisseur.
\end{itemize}

    \subsection{Fonctions du contrôleur}
\paragraph{index} Cette fonction retourne la page d'accueil du module. Elle permet de visualiser immédiatement les produits qui sont actuellement en rupture ainsi que tous les produits que peut proposer le site. Une pagination est effectuée, bien entendu, car il ne serait pas très pratique d'afficher tous les objets sur une même page. La pagination permet de n'afficher que 10 produits à la fois.

\paragraph{Créer/Modifier/Supprimer Fournisseur}

\paragraph{Créer/Modifier/Supprimer Produit}

\paragraph{Commander} Cette fonction permet de commander une certaine quantité d'un produit à son attribut quantiteCommandeFournisseur.

\paragraph{Valider commande} Cette fonction permet de valider une commande. À la page de "Commande" de chaque fournisseur, le site web liste les produits qui sont actuellement en commande, c'est à dire les produits dont l'attribut quantiteCommandeFournisseur est non-nul. Un bouton juste en dessous de cette liste permet de valider la commande : on ajoute la quantité commandé à la quantité en stock.


\begin{figure}[H]
    \centering
    \includegraphics[width=\linewidth]{./logos/gestionStock.png}
    \caption{Commander des produits auprès d'un fournisseur}
\end{figure}

%%%%%%%%%%%%%%%%%%%%%%%%%%%%%%%%%%%%%%%%%%%%%%%%%%%%%%%%%%%%%%%%%%%%%%%%%%%%%%%%%%%%%%%%%%%%%%%%%%%%ù
%%%%%%%%%%%%%%%%%%%%%%%%%%%%%%%%%%%%%%%%%%%%%%%%%%%%%%%%%%%%%%%%%%%%%%%%%%%%%%%%%%%%%%%%%%%%%%%%%%%%ù
%%%%%%%%%%%%%%%%%%%%%%%%%%%%%%%%%%%%%%%%%%%%%%%%%%%%%%%%%%%%%%%%%%%%%%%%%%%%%%%%%%%%%%%%%%%%%%%%%%%%ù
\chapter{Module Ventes}%{{{
% TODO Seb

    \paragraph{}Nous allons vous présenter dans cette partie le module de ventes du site web. Voici tout d'abord une capture d'écran de la page de ventes vue par un kftier.
    \begin{center}
        \includegraphics[width=1\linewidth]{logos/ventes.png}
    \end{center}
    \paragraph{}Nous allons dans un premier temps définir les fonctionnalités disponible dans la page, puis nous détaillerons le fonctionnement de celles-ci.

    \section{Détail des fonctionnalités}

        \paragraph{}Lorsque un kftier veut vendre un produit il se rend sur cette page. La page indique plusieurs informations : \\
            \begin{itemize}
                \item Les produits avec pour chaque produit : son prix, sa quantité restante entre parenthèse, une image le représentant ;\\
                \item Une indication "épuisé" lorsqu'un produit n'est plus disponible ;\\
                \item La dernière vente effectuée et la possibilité d'annuler celle-ci en cas d'erreur ;\\
                \item Une liste, pouvant être dissimuler, permettant de sélectionner la personne voulant payer en dette.\\
            \end{itemize}

    \section{Fonctionnement du module}

        \subsection{Modèle de base de donnée}

            \paragraph{}Les attributs de la table vente, permettant de stocker les informations de vente sont les suivants :\\
            \begin{itemize}
                \item produit\_id : identifiant du produit venu ;\\
                \item date : la date de la vente ;\\
                \item quantite : la quantité vendue ;\\
                \item user\_id : identifiant de l'utilisateur en cas de vente avec dette.
            \end{itemize}

        \subsection{Fonction du contrôleur}

            \paragraph{index}Cette fonction permet d'afficher l'index des ventes. Elle récupère tous les produits à afficher et vérifie leurs disponibilités. Elle récupère de plus la dernière vente effectuée.

            \paragraph{produit\_vente}Cette fonction permet de réaliser une vente. Elle vérifie la présence ou non de la sélection d'une personne et effectue la vente. Dans le cas où une personne est sélectionnée elle augmente la dette de celle-ci du montant de la vente. Les stocks sont de plus déduit en fonction de la quantité sélectionnée.
            \paragraph{annuler\_vente}Cette fonction annule la dernière vente effectuée en supprimant l'entrée de la base de donnée. Le stock est réapprovisionner et si une dette a été ajoutée elle est annulée.

        \subsection{Fonction du template}

            \paragraph{}Le template réalisant l'affichage de la page est formée de plusieurs formulaires et ayant chacun leurs boutons de validation.

            \paragraph{}Du javascript a été nécessaire pour la sélection du nom de la personne à ajouter en dette, en effet une seule sélection est répercutée sur tous les formulaires de la page.

            \paragraph{}Enfin une pagination est mise en place à l'aide de Django et définie à la fois dans le contrôleur et la vue.%}}}

%%%%%%%%%%%%%%%%%%%%%%%%%%%%%%%%%%%%%%%%%%%%%%%%%%%%%%%%%%%%%%%%%%%%%%%%%%%%%%%%%%%%%%%%%%%%%%%%%%%%ù
%%%%%%%%%%%%%%%%%%%%%%%%%%%%%%%%%%%%%%%%%%%%%%%%%%%%%%%%%%%%%%%%%%%%%%%%%%%%%%%%%%%%%%%%%%%%%%%%%%%%ù
%%%%%%%%%%%%%%%%%%%%%%%%%%%%%%%%%%%%%%%%%%%%%%%%%%%%%%%%%%%%%%%%%%%%%%%%%%%%%%%%%%%%%%%%%%%%%%%%%%%%ù
\chapter{Module Administration}
% TODO Loic

Le module d'administration permet de faire beaucoup de choses et n'est pas accessible à tout le monde. Nous n'avons pas
souhaité utiliser le module d'administration de Django car elle n'était pas du tout personnalisée et l'intérêt du projet
aurait été perdu. Nous avons donc souhaité réaliser notre propre module d'administration avec des fonctions plus
avancées que celles proposées par Django.

\begin{figure}[H]
    \centering
    \includegraphics[width=\linewidth]{./logos/administration.png}
    \caption{L'accueil de l'administration}
\end{figure}

Comme on peut le voir sur la figure précédente, l'accueil de l'administration n'est qu'une page permettant d'accéder aux
sous-module d'administration du site. Toutes ces pages ne sont accessibles qu'aux responsables du club Kfet. Ce sont les
responsables du club qui vont avoir la main sur les finances (statistiques de ventes), la gestion des stocks, les
différents menus de la Kfet, les dettes, les types de paiement et enfin les commandes clients.

\section{Les modules d'administration}
    \subsection{Gérer les commandes}
Lorsqu'une personne se connecte sur le site, elle a la possibilité d'acheter des produits. Ces produits sont en fait
ajouter dans son panier. Quand le panier est validé, il passe dans une "commande cliente" et c'est au Kftier ou aux
responsables de la Kfet de prendre la main sur la commande.
\begin{enumerate}
    \item La commande est en premier dans l'état en attente : en attente de validation d'un Kftier ;
    \item Puis la commande passe à "En cours" : La commande est en cours de traitement (On prépare les produits) ;
    \item La commande passe à l'état "Terminée" quand elle a été récupérée par le client et payée.
\end{enumerate}

Un dernier état de commande existe "Annulée", cet état parle de lui-même.


    \subsection{Gérer les dettes}
Les dettes est une particularité très intéressante pour les clients de la Kfet. Comme son nom l'indique, si le client n'a pas d'argent pour payer, on demande de créer un compte qui permet de payer plus tard.

Le client, au moment de valider son panier, a le choix de son mode de paiement : 
\begin{itemize}
    \item Liquide ;
    \item Monéo ;
    \item Dette.
\end{itemize}

Un client ne peut pas dépasser 5 euros de dettes.

    \subsection{Gérer les stocks}
Ce module a été décrit dans le chapitre \ref{stock}.

    \subsection{Gérer les menus}
La Kfet propose des menus le midi afin de payer moins cher un ensemble de produits. Nous avons donc reproduit les menus
sur le site. Le module de gestion des menus permet très facilement de créer des menus composé de :
\begin{itemize}
    \item Un plat (spécifié par le type d'un produit, par exemple pizza) ;
    \item Une boisson ;
    \item et de 1 ou 2 article.
\end{itemize}

    \subsection{Gérer les paiements}
Sur ce module, on peut ajouter/modifier/supprimer les types de paiement proposés aux clients. Chaque type de paiement
possède une option qui précise s'il est disponible aux clients commandant des produits sur le site internet. Tous les
types de paiements qui sont listés sur cette pages sont proposés aussi au comptoir de la Kfet.

    \subsection{Gestion des ventes}
Ce petit module permet d'obtenir le chiffre d'affaire par jour. C'est un module incomplet à ce jour mais, à terme, il y sera possible d'afficher des graphiques (vente à la semaine, au mois) ou encore les statistiques des ventes par produit.


%%%%%%%%%%%%%%%%%%%%%%%%%%%%%%%%%%%%%%%%%%%%%%%%%%%%%%%%%%%%%%%%%%%%%%%%%%%%%%%%%%%%%%%%%%%%%%%%%%%%ù
%%%%%%%%%%%%%%%%%%%%%%%%%%%%%%%%%%%%%%%%%%%%%%%%%%%%%%%%%%%%%%%%%%%%%%%%%%%%%%%%%%%%%%%%%%%%%%%%%%%%ù
%%%%%%%%%%%%%%%%%%%%%%%%%%%%%%%%%%%%%%%%%%%%%%%%%%%%%%%%%%%%%%%%%%%%%%%%%%%%%%%%%%%%%%%%%%%%%%%%%%%%ù
\chapter{Conclusion}
% TODO Loïc

Ce projet nous a permis de découvrir une nouvelle technologie web ainsi que la mise ne place d'un site de vente en ligne et tous les besoins qui vont avec. En plus d'un site de vente en ligne, un gros travail a été effectué sur la partie gestion du club de la Kfet afin d'obtenir une gestion de stock et des ventes qui étaient totalement inexistantes auparavant.

\paragraph{}
Ce site devrait être mis en production pour la rentrée de Septembre 2012 ce qui permettrait au Kfetiers de partir sur de bonnes bases pour les années à venir.

\paragraph{}
Malgré tout le temps passé à développer le site internet, il manque certaines fonctionnalités (historique des commandes fournisseurs, envoi d'alerte aux clients en cas de dépassement de dette)) qui viendraient compléter le site. Nous allons tous les deux continuer à travailler sur ce site bénévolement pour le club Kfet afin qu'ils répondent exactement aux besoins de la Kfet.\\
L'an prochain, des étudiants de Di4 se sont manifestés auprès de nous afin d'en reprendre la maintenance.


%%%%%%%%%%%%%%%%%%%%%%%%%%%%%%%%%%%%%%%%%%%%%%%%%%%%%%%%%%%%%%%%%%%%%%%%%%%%%%%%%%%%%%%%%%%%%%%%%%%%ù
%%%%%%%%%%%%%%%%%%%%%%%%%%%%%%%%%%%%%%%%%%%%%%%%%%%%%%%%%%%%%%%%%%%%%%%%%%%%%%%%%%%%%%%%%%%%%%%%%%%%ù
%%%%%%%%%%%%%%%%%%%%%%%%%%%%%%%%%%%%%%%%%%%%%%%%%%%%%%%%%%%%%%%%%%%%%%%%%%%%%%%%%%%%%%%%%%%%%%%%%%%%ù
%%%%%%%%%%%%%%%%%%%%%%%%%%%%%%%%%%%%%%%%%%%%%%%%%%%%%%%%%%%%%%%%%%%%%%%%%%%%%%%%%%%%%%%%%%%%%%%%%%%%ù
%%%%%%%%%%%%%%%%%%%%%%%%%%%%%%%%%%%%%%%%%%%%%%%%%%%%%%%%%%%%%%%%%%%%%%%%%%%%%%%%%%%%%%%%%%%%%%%%%%%%ù

\end{document}

